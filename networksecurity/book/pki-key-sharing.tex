\graphicspath{{pki-key-sharing}}
\section{Gestione e distribuzione delle chiavi crittografiche}
\paragraph{Gestione delle chiavi crittografiche}L'uso sicuro degli algoritmi di chiave crittografica dipende dalla protezione delle chiavi crittografiche; la loro gestione è il processo di amministrazione o gestione delle chiavi crittografiche per un sistema crittografico: comporta la generazione, la creazione, la protezione, l'archiviazione, lo scambio, la sostituzione e l'utilizzo delle chiavi e consente restrizioni selettive per determinate chiavi. Oltre alla restrizione di accesso, la gestione delle chiavi implica anche il monitoraggio e la registrazione dell'accesso, dell'uso e del contesto di ciascuna chiave. Un sistema di gestione delle chiavi includerà anche server di chiavi, procedure per gli utenti e protocolli. La sicurezza del sistema crittografico dipende dalla gestione efficace delle chiavi. 

\paragraph{Tecnica di distribuzione delle chiavi}Affinché la crittografia simmetrica funzioni, le due parti coinvolte in uno scambio devono condividere la stessa chiave che deve essere protetta dall'accesso di terzi. Inoltre, cambi frequenti della chiave sono solitamente auspicabili per limitare la quantità di dati compromessi se un attaccante riesce a scoprire la chiave. Pertanto, la forza di qualsiasi sistema crittografico dipende dalla tecnica di distribuzione delle chiavi, un termine che si riferisce ai mezzi per consegnare una chiave a due parti che desiderano scambiare dati, senza consentire ad altri di vedere la chiave. 
\paragraph{Distribuzione della chiave simmetrica}Per due parti A e B, la distribuzione della chiave può essere realizzata in diversi modi:
\begin{figure}[thbp]
	\centering
	\includegraphics[width=.7\linewidth]{distribuzione-chiavi-tra-due-entità}
	\caption{opzioni per la distribuzione delle chiavi.}
		\label{fig:distribuzione-chiavi}
\end{figure}
\begin{enumerate}[(a)]
	\item A può selezionare una chiave e consegnarla fisicamente a B;
	\item una terza parte può selezionare la chiave  e consegnarla fisicamente sia ad A che a B;
	\item se A e B hanno precedentemente e recentemente utilizzato una chiave,  una delle parti può trasmettere la nuova chiave all'altra, cifrata utilizzando la vecchia chiave;
	\item se A e B hanno ciascuno una connessione crittografata con una terza parte C, C può consegnare una chiave sui collegamenti crittografati a A e B.
\end{enumerate}
Le opzioni 1 e 2 richiedono la consegna manuale di una chiave. Per la crittografia dei collegamenti, questo è un requisito ragionevole, poiché ogni dispositivo di crittografia dei collegamenti scambierà dati solo con il suo partner all'altro capo del collegamento. Tuttavia, per la crittografia end-to-end su una rete, la consegna manuale è scomoda. In un sistema distribuito, qualsiasi utente  o server può dover partecipare a scambi con molti altri utenti e server nel tempo. Pertanto, ogni endpoint ha bisogno di un certo numero di chiavi fornite dinamicamente. Il problema è particolarmente difficile in un sistema distribuito su una vasta area.

La scala del problema dipende dal numero di coppie comunicanti che devono essere supportate. Se la crittografia end-to-end viene eseguita a livello di rete o IP, allora è necessaria una chiave per ogni coppia di host sulla rete che desidera comunicare. Pertanto, se ci sono $n$ host, il numero di chiavi richieste è: $$\frac{n(n-1)}{2}$$ Se la crittografia viene eseguita a livello di applicazione, allora è necessaria una chiave per ogni coppia di utenti o processi che richiedono comunicazione. Pertanto, una rete può avere centinaia di host ma migliaia di utenti e processi. Una rete che utilizza la crittografia a livello di nodo con 1000 nodi potrebbe concepibilmente dover distribuire fino a mezzo milione di chiavi. Se quella stessa rete supporta 10.000 applicazioni, allora potrebbero essere necessarie fino a 50 milioni di chiavi per la crittografia a livello di applicazione.

L'opzione 3 è una possibilità sia per la crittografia dei collegamenti che per la crittografia end-to-end, ma se un attaccante riesce mai a ottenere l'accesso a una chiave, tutte le chiavi saranno rilevate. Inoltre, la distribuzione iniziale di potenzialmente milioni di chiavi deve ancora essere effettuata. 

Per la crittografia end-to-end, una qualche variazione dell'opzione 4 è stata ampiamente adottata. In questo schema, un centro di distribuzione delle chiavi è responsabile della distribuzione delle chiavi a coppie di utenti (host, processi, applicazioni) secondo necessità. Ogni utente deve condividere una chiave unica con il centro di distribuzione delle chiavi per scopi di distribuzione delle chiavi.

La \cref{fig:distribuzione-chiavi} illustra due diverse opzioni, ciascuna con due variazioni per la distribuzione delle chiavi. I numeri lungo le linee rappresentano i passaggi dello scambio. In questi esempi, esiste una connessione tra le entità A e B, che desiderano scambiare informazioni utilizzando tecniche crittografiche. A questo scopo, necessitano di una chiave di sessione temporanea che durerà per la durata di una connessione logica, come una connessione TCP. A e B condividono ciascuno una chiave master di lunga durata con una terza parte coinvolta nella fornitura della chiave di sessione. La chiave di sessione è etichettata come $K_s$ e le chiavi master tra le entità A e B e la terza parte sono etichettate come $K_{ma}$ e $K_mb$ rispettivamente. Un centro di traduzione chiavi (KTC) trasferisce chiavi simmetriche per future comunicazioni tra due entità, almeno una delle quali ha la capacità di generare o acquisire chiavi simmetriche autonomamente. L'entità A genera o acquisisce una chiave simmetrica da utilizzare come chiave di sessione per la comunicazione con B. A cripta la chiave utilizzando la chiave master che condivide con il KTC e invia la chiave criptata al KTC. Il KTC decripta la chiave di sessione, la cripta nuovamente con la chiave master che condivide con B e invia quella chiave di sessione criptata nuovamente ad A affinché A la inoltri a B, oppure la invia direttamente a B.

\paragraph{KDC (Key Distribution Center)}Un centro di distribuzione chiavi (KDC) genera e distribuisce chiavi di sessione. L'entità A invia una richiesta al KDC per una chiave simmetrica da utilizzare come chiave di sessione per la comunicazione con B. Il KDC genera una chiave di sessione simmetrica, quindi la cripta con la chiave master che condivide con B e la invia a B. In alternativa, invia entrambi i valori di chiave criptati ad A, e A inoltra la chiave di sessione criptata con la chiave master condivisa dal KDC e B a B.

Vi sono un certo numero di dettagli omessi, ad esempio, le parti che scambiano chiavi devono autenticarsi reciprocamente. I timestamp sono spesso utilizzati per limitare il tempo in cui può avvenire uno scambio di chiavi e/o la durata di vita di una chiave scambiata.