

\documentclass{report}
\usepackage[italian]{babel}
\usepackage{algorithm}
\usepackage{algpseudocode}
\usepackage{soul}
\usepackage{amsmath}
\usepackage{amssymb}
\usepackage{graphicx}
\usepackage{geometry}
\usepackage{xspace}
\newtheorem{definition}{Definizione}
\newtheorem{theorem}{Teorema}
\newtheorem{lemma}{Lemma}
\newtheorem{proof}{Dimostrazione}
\begin{document}
\title{Crittografia}
\author{Riccardo Torre}
\chapter*{Prefazione} Questo report è una riscrittura del lavoro dell'Egregio Dottor Michele Perlotto. 
\begin{quotation}
	\textit{``I Procioni sono meglio dei Panda Rossi'' }-- cit. Michele Perlotto.
\end{quotation}
\date{}
\maketitle

\tableofcontents % Output the table of contents


\input{chapters/chapter1.tex}
\input{chapters/chapter2 - crittogradia in cascata.tex}
\input{chapters/chapter3 - cenni alla teoria dei numeri.tex}
\input{chapters/chapter4 - crittografia a chiave pubblica.tex}
\input{chapters/chapter5 - crittografia a blocchi.tex}
\input{chapters/chapter6 - codifica di un singolo bit.tex}
\input{chapters/chapter7 - codifica di sequenze di bit.tex}
\input{chapters/chapter8 - lancio della moneta e hard core predicate.tex}
\input{chapters/chapter9 - generatori pseudocasuali.tex}
\input{chapters/chapter10 - funzioni pseudocasuali.tex}
\input{chapters/chapter11 - autenticazione di messaggi.tex}
\input{chapters/chapter12 - protocolli vari.tex}
\input{chapters/chapter13 - zero knowledge.tex}
\end{document}
